\input dantemacs
\documentstyle[11pt]{article}
\begin{document}
\begin{titlepage}
	\begin{center}
		\vspace{1in}
		{\LARGE \it The Dartmouth Dante Project} \\
		\vspace{.3in}
		{\large Technical Memorandum} \\
		\vspace{2in}
		{\large \bf DDP Standard Escape Sequences} \\
		\vspace{.2in}
		{\it Stephen Campbell} \\
		\vfill
		\today
	\end{center}
\end{titlepage}

\pagestyle{headings}
\section{Introduction}
This memo describes the character escape sequence that will be used in
the Dartmouth Dante Project (DDP) database.
This revision of the memo reflects the shift from the troff text
processing system to the \TeX\ system.
The most significant effect of this change is the new handling of
Greek characters.

An escape sequence is a sequence of special characters that appears within a
file of printable text such as the Dante commentaries.
The escape sequence characters are special in that they do not appear in
the final displayed version of the text.  Instead they indicate the
occurrence of unusual characters or of changes in the character font.
In order to have their intended effect, escape sequences must be recognized
by a program and converted into the intended character representation.
Standard escape sequences are used because the intended representation -
{\bf boldface}, for example -
is produced differently for different terminals and printers.
Further, there is no guarantee that a given escape sequence will produce
the desired result on all display devices;
some devices cannot generate all the special characters.

In the Dante Project, 
we will use these sequences in the text that we load into the database.
When we retrieve text from the database for display
on terminals or for typesetting, the software will derive
the needed special characters from these standard sequences.

To the greatest extent possible, the Kurtzweil Data Entry Machine
(KDEM) should generate the standard character sequences.
Where the KDEM cannot conveniently generate them, we will
provide Unix programs to translate KDEM characters into the standard
sequences.

\section{Escape Sequences for Special Characters}
Currently all escape sequences fall into one of two categories.
One type is used to represent unusual characters such as accented European
characters and Greek characters.
The other type is used to cause a change in font.

\subsection{Diacritics}
The following table lists the Dante standard diacritic
character escapes.
These special characters should be followed
by a regular character that is to receive the diacritic mark.
\begin{center}
  \begin{tabular}{lll}
    {\em Character} & {\em Effect} & {\em Example} \\ \hline
    At Sign (@)     & Accent grave & {\tt @a} gives \`{a} \\
    Dollar sign (\$) & Accent acute & {\tt \$a} gives \'{a} \\
    Comma (,)       & Umlaut       & {\tt ,u} gives \"{u} \\
    Pound (\#)	    & Circumflex   & {\tt \#o} gives \^{o} \\
    Dollar (\$)     & Cedilla      & {\tt \$c} gives \c{c} \\
  \end{tabular}
\end{center}

\subsection{European Quotes}
The following table shows the escape sequences we will use to generate
European quote marks.
\begin{center}
  \begin{tabular}{lll}
    {\em Character} & {\em Effect} & {\em Example} \\ \hline
    Less than ($<$)   & Left quote   & {\tt $<$} gives \< \\
    Greater than ($>$) & Right quote & {\tt $>$} gives \> \\
  \end{tabular}
\end{center}

\subsection{Ligatures}
The following table lists Dante ligature escapes.
These marks are used before the letters {\it ae} or {\it oe}
(or the capitalized equivalent)
to represent the ligature form of those letters.
\begin{center}
  \begin{tabular}{lll}
    {\em Character} & {\em Effect} & {\em Example} \\ \hline
    Double At Sign (@@) & Ligature & {\tt @@ae} gives \ae \\
  \end{tabular}
\end{center}

\subsection{Greek Letters}
Greek letters are represented by a percent sign (\%)
followed by the name
of the letter.
For an uppercase Greek letter, capitalize the first letter of the name.
If the uppercase Greek letter is the same as its Roman equivalent, as in
the uppercase alpha, then there is no special representation; we just use
the Roman letter.
We also use a plain ``o'' for omicron.

The following table lists all the Greek letters that we can
represent.\footnote{These names are the ones used in the \TeX\ typesetting
system.
See Donald E. Knuth, {\it The \TeX book,}  Addison-Wesley, 1984.}
Some are variants of other letters, and their names begin with {\it var}.
\begin{center}
  \begin{tabular}{llll}
    \multicolumn{4}{c}{\em Lowercase} \\
    $\alpha$ {\tt \%alpha} & $\theta$ {\tt \%theta} &
	o o & $\tau$ {\tt \%tau} \\
    $\beta$ {\tt \%beta} & $\vartheta$ {\tt \%vartheta} &
	$\pi$ {\tt \%pi} & $\upsilon$ {\tt \%upsilon} \\
    $\gamma$ {\tt \%gamma} & $\iota$ {\tt \%iota} &
	$\varpi$ {\tt \%varpi} & $\phi$ {\tt \%phi} \\
    $\delta$ {\tt \%delta} & $\kappa$ {\tt \%kappa} &
	$\rho$ {\tt \%rho} & $\varphi$ {\tt \%varphi} \\
    $\epsilon$ {\tt \%epsilon} & $\lambda$ {\tt \%lambda} &
	$\varrho$ {\tt \%varrho} & $\chi$ {\tt \%chi} \\
    $\varepsilon$ {\tt \%varepsilon} & $\mu$ {\tt \%mu} &
	$\sigma$ {\tt \%sigma} & $\psi$ {\tt \%psi} \\
    $\zeta$ {\tt \%zeta} & $\nu$ {\tt \%nu} &
	$\varsigma$ {\tt \%varsigma} & $\omega$ {\tt \%omega} \\
    $\eta$ {\tt \%eta} & $\xi$ {\tt \%xi} \\ \\
    \multicolumn{4}{c}{\em Uppercase} \\
    $\Gamma$ {\tt \%Gamma} & $\Lambda$ {\tt \%Lambda} &
	$\Sigma$ {\tt \%Sigma} & $\Psi$ {\tt \%Psi} \\
    $\Delta$ {\tt \%Delta} & $\Xi$ {\tt \%Xi} &
	$\Upsilon$ {\tt \%Upsilon} & $\Omega$ {\tt \%Omega} \\
    $\Theta$ {\tt \%Theta} & $\Pi$ {\tt \%Pi} & $\Phi$ {\tt \%Phi} \\
  \end{tabular}
\end{center}

\section{Font Changes}
This section lists escape sequences that we will use to cause changes
in the font in which characters are displayed.
Currently we have four possible fonts: Roman (plain), italics, boldface, and
superscript characters.
Roman font is used for normal text, for example this sentence.
The escapes can be used to change to another font and then
change back to Roman.
\begin{center}
  \begin{tabular}{lll}
    {\em Character} & {\em Effect} & {\em Example} \\ \hline
    Vertical bar ($|$) & Start Boldface &
	{\tt $|$Bold\tilde} gives {\bf Bold} \\
    Caret (\caret) & Start Italics &
	{\tt \caret Italics\tilde} gives {\it Italics} \\
    Plus (+) & Start Superscript & {\tt 1+st.\tilde} gives $1^{st.}$ \\
    Tilde (\tilde) & Return to Roman & See above. \\
  \end{tabular}
\end{center}
\end{document}
